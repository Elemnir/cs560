%%%%%%%%%%%%%%%%%%%%%%%%%%%%%%%%%%%%%%%%%
% Journal Article
% LaTeX Template
% Version 1.3 (9/9/13)
%
% This template has been downloaded from:
% http://www.LaTeXTemplates.com
%
% Original author:
% Frits Wenneker (http://www.howtotex.com)
%
% License:
% CC BY-NC-SA 3.0 (http://creativecommons.org/licenses/by-nc-sa/3.0/)
%
%%%%%%%%%%%%%%%%%%%%%%%%%%%%%%%%%%%%%%%%%

%----------------------------------------------------------------------------------------
%	PACKAGES AND OTHER DOCUMENT CONFIGURATIONS
%----------------------------------------------------------------------------------------

\documentclass[twoside]{article}

\usepackage{lipsum} % Package to generate dummy text throughout this template

\usepackage[sc]{mathpazo} % Use the Palatino font
\usepackage[T1]{fontenc} % Use 8-bit encoding that has 256 glyphs
\linespread{1.05} % Line spacing - Palatino needs more space between lines
\usepackage{microtype} % Slightly tweak font spacing for aesthetics

\usepackage[hmarginratio=1:1,top=32mm,columnsep=20pt]{geometry} % Document margins
\usepackage{multicol} % Used for the two-column layout of the document
\usepackage[hang, small,labelfont=bf,up,textfont=it,up]{caption} % Custom captions under/above floats in tables or figures
\usepackage{booktabs} % Horizontal rules in tables
\usepackage{float} % Required for tables and figures in the multi-column environment - they need to be placed in specific locations with the [H] (e.g. \begin{table}[H])
\usepackage{hyperref} % For hyperlinks in the PDF

\usepackage{lettrine} % The lettrine is the first enlarged letter at the beginning of the text
\usepackage{paralist} % Used for the compactitem environment which makes bullet points with less space between them

\usepackage{abstract} % Allows abstract customization
\renewcommand{\abstractnamefont}{\normalfont\bfseries} % Set the "Abstract" text to bold
\renewcommand{\abstracttextfont}{\normalfont\small\itshape} % Set the abstract itself to small italic text

\usepackage{titlesec} % Allows customization of titles
\renewcommand\thesection{\Roman{section}} % Roman numerals for the sections
\renewcommand\thesubsection{\Roman{subsection}} % Roman numerals for subsections
\titleformat{\section}[block]{\large\scshape\centering}{\thesection.}{1em}{} % Change the look of the section titles
\titleformat{\subsection}[block]{\large}{\thesubsection.}{1em}{} % Change the look of the section titles

\usepackage{fancyhdr} % Headers and footers
\pagestyle{fancy} % All pages have headers and footers
\fancyhead{} % Blank out the default header
\fancyfoot{} % Blank out the default footer
\fancyhead{} % Custom header text
\fancyfoot[RO,LE]{\thepage} % Custom footer text

%----------------------------------------------------------------------------------------
%	TITLE SECTION
%----------------------------------------------------------------------------------------

\title{\vspace{-15mm}\fontsize{24pt}{10pt}\selectfont\textbf{Programming
Assignment 1: Filesystem}} % Article title

\author{
\large
\textsc{Ben Olson and Adam Howard} \\[2mm] % Your name
\normalsize University of Tennessee, Knoxville \\ % Your institution
\vspace{-5mm}
}
\date{}

%----------------------------------------------------------------------------------------

\begin{document}

\maketitle % Insert title

\thispagestyle{fancy} % All pages have headers and footers

%----------------------------------------------------------------------------------------
%	ABSTRACT
%----------------------------------------------------------------------------------------

\begin{abstract}

  The purpose of this assignment is to study the functionality of a filesystem,
  and to support this knowledge with a practical application of one. In
  addition, a simple shell should be implemented such that the filesystem has a
  convenient user interface, as well as a method of remotely connecting to said
  shell. These three parts of the program combine to give the student a
  comprehensive understanding about how a simple remote connection server,
  shell, and filesystem work.

\end{abstract}

%----------------------------------------------------------------------------------------
%	ARTICLE CONTENTS
%----------------------------------------------------------------------------------------

\begin{multicols}{2} % Two-column layout throughout the main article text

\section{Introduction}

\lettrine[nindent=0em,lines=3]{T} his program can be separated into three parts,
for the sake of dividing the work
evenly between a group of two or three students. The first is the filesystem,
which creates a file on disk that represents a disk, and can be written to and
read from by a user.  The second is the shell, which is the interface through
which the user communicates with the filesystem, and which simply reads text
from \texttt{stdin} and calls functions that implement the filesystem.  The third and
final section is an implementation of a remote shell, which simply allows
connections on some socket and connects the remote user to the shell on the
server machine.  This design document will first describe the compilation and
usage of the project, and then describe in detail the implementations of each of
the parts.

%------------------------------------------------

\section{Compilation}

This program was intended to be run on a Linux system, although it will work on
any system that supports \texttt{make} and POSIX sockets, and contains a modern
C++ compiler, with perhaps minor modifications. To begin the compilation
process, let us first examine the layout of the source directories:

\begin{description}
  \item[\texttt{server}] contains the single source file for the server
    application.
  \item[\texttt{client}] contains the single source file for the client
    application.
  \item[\texttt{fakefs}] contains the header and implementation of the
    filesystem, as well as the shell.
\end{description}

To compile the complete application, one must simply use the provided GNU
\texttt{Makefile}, by typing \texttt{make} into their shell. By default, this will compile the
server, client, filesystem, and shell, and put one executable into each of the
three source directories. To delete these executables and compile for a separate
system, one may use the \texttt{make clean} command.

%------------------------------------------------

\section{Usage}

The usage of the program is rather simple. For the sake of testing the program,
one can simply run the \texttt{run.sh} script. This is a rather simple script
that first compiles the programs (using \texttt{make} as described above),
starts the server on port \texttt{30000} in the background, and then runs the
client on the same port so that it connects to the server. This will bring up a
shell with the required commands implemented.
\\
To use the program manually, one must first start the \texttt{server} program,
with the \texttt{fakefs} executable in the user's \texttt{PATH} variable. The
one argument that this program takes is the port number.  Next, execute the
\texttt{client} executable on the same port, so that it connects to the server.
Once connected successfully, begin typing commands on \texttt{stdin}.

%------------------------------------------------

\section{Implementation}

In this section we will describe in detail the implementation of each of the
three parts of the project, starting from the top and working down to the
low-level workings of the filesystem.

\subsection{Remote connection}

The \texttt{server} and \texttt{client} source directories contain the source
code for this part of the project, in \texttt{server.c} and \texttt{client.c}
respectively. First, the server accepts a port number as an argument. If this
argument is valid, it binds to that socket and begins to listen for clients. At
this point, it cannot continue until the client program is run on the same
socket. Once the client connects (using the \texttt{connect()} system call), the
server continues by accepting the connection and forking off a new process which
will run the shell. This process uses \texttt{dup2()} to forward all
\texttt{stdout} from the shell to the socket (and consequently to the client),
and gets all \texttt{stdin} from the socket as well. It uses \texttt{execvp} to
execute the shell.
\\
The client is a bit more complicated, simply because it has to handle receiving
text from the socket while also sending \texttt{stdin} to the socket. Normally,
this might warrant a multiprocess solution, however, using the \texttt{select()}
system call is entirely sufficient. This function creates a set of all file
descriptors that can have text flowing through them, and selects the first one
that has something coming from it. This way, the client can simply loop
infinitely, select the file descriptor that has activity, and pipe input from
\texttt{stdin} if no output is available to be received. This way, the user can
simultaneously input text into the shell while receiving its output
successfully.

\subsection{Shell}

This is by far the simplest of the three sections. Implemented in
\texttt{fakefs/main.cpp}, it simply uses C++ stringstreams to read from
\texttt{stdin} and output the results of the filesystem to \texttt{stdout}.
Since this is the program that is launched by the server, the \texttt{stdin}
comes from the socket that the client is connected to, and the \texttt{stdout}
goes to the socket that is read by the client.
\\
When it receives a valid command, the shell calls the corresponding function in
the filesystem.

\subsection{Filesystem}

The filesystem is the main purpose of the project, while the other parts are
centered around the usage of the filesystem itself. Implemented in
\texttt{fakefs/filesystem.cpp}, it includes functions for the shell to call to
do a variety of modifications to the virtual disk. The following higher-level
functions are implemented in the \texttt{Filesystem} class:

\begin{description}
  \item[\texttt{load}] checks to see if the virtual disk already exists (in the
    filename \texttt{.disk.ffs}), and simply opens it if it does. Else, it
    creates the filesystem and prompts the user to use the \texttt{mkfs} command
    to initialize the filesystem.
  \item[\texttt{}] end
\end{description}

%----------------------------------------------------------------------------------------

\end{multicols}

\end{document}
